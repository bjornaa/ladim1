\documentclass[a4paper]{article}
% generated by Docutils <http://docutils.sourceforge.net/>
\usepackage{fixltx2e} % LaTeX patches, \textsubscript
\usepackage{cmap} % fix search and cut-and-paste in Acrobat
\usepackage{ifthen}
\usepackage[T1]{fontenc}
\usepackage[utf8]{inputenc}

%%% Custom LaTeX preamble
% PDF Standard Fonts
\usepackage{mathptmx} % Times
\usepackage[scaled=.90]{helvet}
\usepackage{courier}

%%% User specified packages and stylesheets

%%% Fallback definitions for Docutils-specific commands

% titlereference role
\providecommand*{\DUroletitlereference}[1]{\textsl{#1}}

% hyperlinks:
\ifthenelse{\isundefined{\hypersetup}}{
  \usepackage[colorlinks=true,linkcolor=blue,urlcolor=blue]{hyperref}
  \urlstyle{same} % normal text font (alternatives: tt, rm, sf)
}{}
\hypersetup{
  pdftitle={Output format},
}

%%% Title Data
\title{\phantomsection%
  Output format%
  \label{output-format}}
\author{}
\date{}

%%% Body
\begin{document}
\maketitle

The particle distributions are written to netCDF files.

===Litt diskusjon om ulik tankegang for partikkelfordelinger

Different formats may be available in the future
(in particular ROMS float format)


%___________________________________________________________________________

\section*{\phantomsection%
  LADIM format%
  \addcontentsline{toc}{section}{LADIM format}%
  \label{ladim-format}%
}

This is a modfication of the LADIM 2 format.
It is not backwards compatible, but scripts should be
easy to modify. The change is done to follow the CF-standard
as far as possible, and to increase flexibility.

The format is close to the format of ``Indexed ragged array
representation of trajectories'' as described in appendix H.4.4 in the
CF-documentation version 1.6.
\url{http://cf-pcmdi.llnl.gov/documents/cf-conventions/1.6/cf-conventions.html\#idp8399648}
There are however modifications, as we are more interested in the
distribution of the particles at a fixed time than to follow the
individual trajectories.

The dimensions are \texttt{time} and \texttt{particle\_dim}. NetCDF 3 format allows only
one unlimited dimension. As the duration and output frequency are
known at the start of a ladim simulation, the particle dimension may
be unlimited.

{[}===LARMOD. For larmod som kjøres fra symbioses-grensesnitt, larmod vet ikke
nødvendigvis hvor lang simuleringen blir. Bruker derfor NetCDF-4 og
begge dimensjoner ubegrenset{]}

The arrays \texttt{pstart(time)} and \texttt{pcount(time)} are used to adress the
particle distributions. For a particle variable, for instance
\texttt{X(particle\_dim)}, the values at time \texttt{t} is found as (python index
notation):
%
\begin{quote}{\ttfamily \raggedright \noindent
X{[}pstart{[}t{]}~:~pstart{[}t{]}+pcount{[}t{]}{]}
}
\end{quote}


%___________________________________________________________________________

\section*{\phantomsection%
  Particle identifier%
  \addcontentsline{toc}{section}{Particle identifier}%
  \label{particle-identifier}%
}

The particle identifier, \texttt{pid} should always be present in the output
file. It is a particle number, starting with 1 and increasing as the
particles are released. The \texttt{pid} follows the particle and is not
reused if the particle becomes inactive.  In particular, \texttt{max(pid)} is
the total number of particles involved in the simulation and may be
larger than the number of active particles at a given time. It also
has the property that if a particle is released before another
particle, it has lower \texttt{pid}.

The particles identifiers at a given time frame \DUroletitlereference{n} can be found from the
output file as \texttt{pid\_n = pid{[}pstart{[}n{]}:pstart{[}n{]}+pcount{[}n{]}{]}}. It has
the following properties:
%
\begin{quote}{\ttfamily \raggedright \noindent
-~pid\_n~is~a~sorted~integer~array\\
-~pid\_n{[}p{]}~=~pid{[}pstart{[}n{]}+p{]}~>=~p+1~with~equality\\
~~~if~all~earlier~particles~are~active~at~time~frame~n.
}
\end{quote}


%___________________________________________________________________________

\section*{\phantomsection%
  Example CDL%
  \addcontentsline{toc}{section}{Example CDL}%
  \label{example-cdl}%
}
%
\begin{quote}{\ttfamily \raggedright \noindent
netcdf~larmod\_out~\{\\
dimensions:\\
~~~~~~particle\_dim~=~UNLIMITED~;~//~(868~currently)\\
~~~~~~time~=~UNLIMITED~;~//~(217~currently)\\
variables:\\
~~~~~~double~time(time)~;\\
~~~~~~~~~~~~~~time:long\_name~=~"time"~;\\
~~~~~~~~~~~~~~time:units~=~"seconds~since~1970-01-01~00:00:00"~;\\
~~~~~~int~pstart(time)~;\\
~~~~~~~~~~~~~~pstart:long\_name~=~"start~index~for~particle~distribution"~;\\
~~~~~~int~pcount(time)~;\\
~~~~~~~~~~~~~~pcount:long\_name~=~"number~of~particles"~;\\
~~~~~~int~pid(particle\_dim)~;\\
~~~~~~~~~~~~~~pid:long\_name~=~"particle~identifier"~;\\
~~~~~~float~lon(particle\_dim)~;\\
~~~~~~~~~~~~~~lon:long\_name~=~"longitude"~;\\
~~~~~~~~~~~~~~lon:units~=~"degrees\_east"~;\\
~~~~~~float~lat(particle\_dim)~;\\
~~~~~~~~~~~~~~lat:long\_name~=~"latitude"~;\\
~~~~~~~~~~~~~~lat:units~=~"degrees\_north"~;\\
~\\
//~global~attributes:\\
~~~~~~~~~~~~~~:history~=~"2013-05-10:~created~by~LARMOD"~;\\
\}
}
\end{quote}

\end{document}
